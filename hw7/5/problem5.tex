\documentclass[oneside]{homework} %%Change `twoside' to `oneside' if you are printing only on the one side of each sheet.
\usepackage{setspace} 
\usepackage{algorithm} 
\usepackage{algorithmic}
\usepackage{multirow}
\usepackage{multicol}
\usepackage{subfigure}
\usepackage{epsfig}
\setlength{\parindent}{0cm}
\renewcommand{\algorithmicrequire}{\textbf{Require:}}
\renewcommand{\algorithmicensure}{\textbf{Iteration:}}
\renewcommand{\algorithmiclastcon}{\textbf{Output:}}
\studname{Cheng Liu}
\collaborator{Introduction to algorithms\\}
\coursename{Analysis of algorithms I}
\hwNo{7}
\uni{cl3173}
\cuni{3rd edition}
\prNo{5}
\begin{document}
\maketitle
\newpage
\section {Problem 35.1-5} 
No, the relationship between Vertex-cover and Clique cannot ensure that there is a constant ratio approximation for the MAX-CLIQUE problem. \\
Recall the relationship, if for certain $G=(V,E)$, we have a size-k clique, then we can ensure that in the graph G'(which is the complement of G), there will be a size $|V|-k$ vertex cover, then we just construct the following approximation algorithm based on this relationship.\\
0. Given a certain $G=(V,E)$ \\
1. Construct the complement of G and name it G'.\\
2. Given G' as input, we run APPRO-VERTEX-COVER algorithm, it will return the answer C. \\
3. Return V-C (C is a vertex cover in G', then V-C is a clique in G, for there must be no edge between any two elements in V-C in G', otherwise C won't be a vertex cover )\\


We use $C_{*}$ to denote the optimal solution for MIN-VERTEX-COVER(G'), then from the textbook, we know the approximation ratio is 2, that is : 
$$|C_{*}| \leq |C| \leq 2|C_{*}|$$
What's more, MAX-CLIQUE is a maximization problem, then we should consider the approximation ratio as $\frac{|V|-|C_{*}|}{|V|-|C|}$, and use the approximation ratio in APPRO-VERTEX-COVER, we have:
\begin{eqnarray*}
  \begin{split}
	\frac{|V|-|C_{*}|}{|V|-|C|} \leq & \frac{|V|-|C_{*}|}{|V|-|2C_{*}|} \\
	 = & 1 + \frac{C_{*}}{|V|-|2C_{*}|} \\
  \end{split}
\end{eqnarray*}
And from the formula above, we could see that the approximation ratio isn't a constant but a function with regard to $|V|$ and the size of optimal vertex cover $|C_{*}|$. \\
Then the conclusion isn't right.

\end{document}
