\documentclass[oneside]{homework} %%Change `twoside' to `oneside' if you are printing only on the one side of each sheet.
\usepackage{setspace} 
\usepackage{algorithm} 
\usepackage{algorithmic}
\usepackage{multirow}
\renewcommand{\algorithmicrequire}{\textbf{Require:}}
\renewcommand{\algorithmicensure}{\textbf{Iteration:}}
\renewcommand{\algorithmiclastcon}{\textbf{Output:}}
\setlength{\parindent}{0cm}
\studname{Cheng Liu}
\collaborator{Introduction to algorithms\\}
\coursename{Analysis of algorithms I}
\hwNo{7}
\uni{cl3173}
\cuni{3rd edition}
\prNo{1}
\begin{document}
\maketitle
\newpage
\section*{Exercise 34.2-5}

At first we see that a language L belongs to NP if and only if there exist a two-input polynomial-time algorithm A and a constant $c$ such that \\
L = $\{ x \in \{0,1\}^{*} : \text{there exists a certificate } y \text{with } |y| = O(|x|^{c}) \text{such that } A(x,y) = 1\}$\\

That is for each language L in NP and each item $x \in \{0,1\}^{*}$, if $ x \in L$, we test all the possible $y$ ($|y| = O(|x|^{c}$),at least we will find a certain $y$ that makes $A(x,y) = 1$. \\

And recall that we say a Language L is decided by a single-input algorithm A' if and only if for every $x \in L, A(x) = 1$, and $x \notin L, A(x) = 0$. \\

Then we can build $A'(x)$ from $A(x,y)$, for each $x \in \{0,1\}^{*}$, we traverse all the possible $y \in \{0,1\}^{*}$ to test $A(x,y)$, if $A(x,y) = 1$, we just return $A'(x) = 1$, else we return 0.\\

Then for each $x$, $|y| = O(|x|^{c})$, all possible $y$ will be $2^{|y|} = 2^{O(|x|^{c})}$, the running time for $A(x,y)$ is polynomial time with regard to input, that is $O(|x|^{z})$ then running time for $A'(x)$ is $O(|x|^{z}2^{O(|x|^{c})}) = 2^{O(|X|^{k_{0}})}$\\

That is $2^{O(n^{k})}$, $k$ is some constant.

The pseudo-code will be:

0.A'(x)\\
1.for each $y \in \{0, 1\}^{*}$ with $|y| = O(|x|^{c})$\\
2.\quad if $(A(x, y) == 1)$ \\
3.\qquad return 1; \\
4.\quad return 0;


\end{document}
