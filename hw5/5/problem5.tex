\documentclass[oneside]{homework} %%Change `twoside' to `oneside' if you are printing only on the one side of each sheet.
\usepackage{setspace} 
\usepackage{algorithm} 
\usepackage{algorithmic}
\usepackage{multirow}
\usepackage{multicol}
\usepackage{subfigure}
\usepackage{epsfig}
\renewcommand{\algorithmicrequire}{\textbf{Require:}}
\renewcommand{\algorithmicensure}{\textbf{Iteration:}}
\renewcommand{\algorithmiclastcon}{\textbf{Output:}}
\studname{Cheng Liu}
\collaborator{Introduction to algorithms\\}
\coursename{Analysis of algorithms I}
\hwNo{5}
\uni{cl3173}
\cuni{3rd edition}
\prNo{5}
\begin{document}
\maketitle
\newpage
\section {Problem 22.5-3} 
We can use a simple counter example to illustrate the mistake of this algorithm.
\begin{figure}[h]
  \centering
  \epsfig{figure=counterexample_2253_1.eps,scale = 1}
  \caption{Counter Example}
  \label{fig:ce}
\end{figure}

From Figure \ref {fig:ce}, we can see there are two strongly connected components,\{w,u,v,x\}, and \{y\}.

By the first DFS, we can get the Table \ref{tlb:ce}:

\begin{table}[h]
\centering
  \begin{tabular}{c|c|c}
	& d & f \\
	\hline
	w & 1 & 10 \\
	\hline
	u & 2 & 9 \\
	\hline
	v & 3 & 6 \\
	\hline
	x & 4 & 5 \\
	\hline
	y & 7 & 8 \\
  \end{tabular}
\caption{First DFS}
\label{tlb:ce}
\end{table}

\begin{table}[h]
\centering
  \begin{tabular}{c|c|c}
	& d & f \\
	\hline
	w & 2 & 9 \\
	\hline
	u & 3 & 8 \\
	\hline
	v & 4 & 5 \\
	\hline
	x & 1 & 10 \\
	\hline
	y & 6 & 7 \\
  \end{tabular}
  \caption{Second DFS}
  \label{tlb:ce2}
\end{table}



If we just re-do the DFS of original graph by the increasing order of the finishing time, the result will be in Table \ref{tlb:ce2}:


And we can see from Table \ref{tlb:ce2}, the second DFS cannot separate the two strongly connected components, thus the algorithm given by Professor Bacon isn't right. The problem is that select the remaining vertex $\mathit{i}$ in the strongly connected components C with minimal finishing time cannot ensure that all the vertices of other strongly connected components that there are edges leave C have been visited at that time.



\section {Problem 22.5-5} 
At first we use the algorithm in Introduction to Algorithm 3rd Edition Page 617 to compute the strongly connected components, the time complexity will be O(V+E). 

And from that algorithm, each vertex belongs to a certain strongly connected components, we number each strongly connected components of a integer from 1 to $|V|$, then for each vertex $\mathit{u}$, it will be associated with a certain integer as a additional property,denoting as $\mathit{u}.no$. 

Traverse the whole V, construct $V_{scc}$ of each distinct $\mathit{u}.no$. 

To prevent duplication to record each $\mathit{u}.no$, we set a array FlagV[1..$|V|$] with all its elements are 0 after initialization to indicate whether this strongly connected component has been added to the $V_{scc}$, in the first time we see a certain $\mathit{u}.no$, we set the corresponding element in FlagV$[\mathit{u}.no]$ to be 1, and add this $\mathit{u}.no$ to $V_{scc}$, when we see $\mathit{u}.no$ again we just ignore it and go on.

The time complexity will also be O(V).

After constructing $V_{scc}$, we construct another set called $E_{scc}$ by traversing the whole E. For each $(\mathit{u},\mathit{y}) \in E$, we add its corresponding $(\mathit{u}.no,\mathit{y}.no)$ to $E_{scc}$ only:

1. It is the edge between to two strongly connected components and 

2. It is the first edge between these two strongly connected components.

To achieve this, we set a $|V|*|V|$ matrix called FlagE with all its elements are 0 after initialization,\textbf{by memory manipulation method, we can finish initialization in O(1)}. 

For a certain $(\mathit{u},\mathit{y}) \in E$, we consider $(\mathit{u}.no,\mathit{y}.no)$, 

If $\mathit{u}.no \neq \mathit{y}.no$ and $FlagE[\mathit{u}.no][\mathit{y}.no] \neq 1$, we add $(\mathit{u}.no,\mathit{y}.no)$ to $E_{scc}$, and set $FlagE[\mathit{u}.no][\mathit{y}.no]$ to 1, else we just ignore this edge.

The time complexity will also be O(E) for traversing the whole E.

If no such O(1) memory manipulation method exists in our problem environment, we can change E to E' by transfer $(\mathit{u},\mathit{y})$ to$(\mathit{u}.no,\mathit{y}.no)$, remove all the self-loops and edge within one strongly connected component, then we apply radix sort to the remaining E',costing O(E+V)(if counting sort used in every component of $(\mathit{u}.no,\mathit{y}.no)$).

For this sorted remaining E', we add the first element to the $E_{scc}$, then traverse the array, if the element is different from the previous element, then we add it to $E_{scc}$. The time complexity of this alternative method is also O(E+V) 

Then we can finish the whole algorithm in O(E+V). 

The correctness of the algorithm is easy to prove. First we compress all the vertices of the same strongly connected components into one vertex in constructing $V_{scc}$. 

Then we only consider those edges between two strongly connected components,and by FlagE, we only add one vertices for each two strongly connected components in constructing $E_{scc}$.


\end{document}
