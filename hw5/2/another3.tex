\documentclass[oneside]{homework} %%Change `twoside' to `oneside' if you are printing only on the one side of each sheet.
\usepackage{setspace} 
\usepackage{algorithm} 
\usepackage{algorithmic}
\usepackage{multirow}
\renewcommand{\algorithmicrequire}{\textbf{Require:}}
\renewcommand{\algorithmicensure}{\textbf{Iteration:}}
\renewcommand{\algorithmiclastcon}{\textbf{Output:}}
\studname{Cheng Liu}
\collaborator{Introduction to algorithms\\}
\coursename{Analysis of algorithms I}
\hwNo{2}
\uni{cl3173}
\cuni{3rd edition}
\prNo{2}
\begin{document}
\maketitle
\newpage
\section{Problem 8.1-3}
Every comparison sort can be drawn as a binary tree, and all possible sequences are equal to the leaves of the tree. If there exist a linear time algorithm, the height of the binary tree must be 
\begin{math}
  cn
\end{math}.
and c is a constant.
\subsection{$\frac{n!}{2}$}

Assume the linear comparison algorithm exists for the number of all possible sequences (denoting $p_{1}$) have Equation \ref{eq:nf2}:
\begin{equation}
  \label{eq:nf2}
  p_{1}\geq\frac{n!}{2} 
\end{equation}

As we all know, a full binary tree with height $h$ has the most leaves among the all the height-$h$ binary trees, and the maximum of the leaves is $2^{h}$,thus to keep the comparison algorithm linear we must have the Equation \ref{eq:hvsn}:  
\begin{equation}
  2^{cn} \geq 2^{h} \geq \frac{n!}{2}
  \label{eq:hvsn}
\end{equation}

Use $lg$ function on both sides, then we'll get the induction \ref{eqa:nf2}:
\begin{eqnarray}
  \begin{split}
	2^{cn} \geq& n!/2 \\
	cn \geq& lg(n!) - 1 \\
	cn \geq& \int_{1}^{n-1}lg(x)dx -1 \\
	cn \geq& (n-1)lg(n-1) - 2n+2-1 \\
	(c+2)n \geq& (n-1)lg(n-1) ~ ~ ~  (n\geq2)\\
	(c+2)n \geq& (n/2)lg(n/2) \\
	(2c+5) \geq& lg(n)
  \end{split}
  \label{eqa:nf2}
\end{eqnarray}

From \ref{eqa:nf2}, we know that c is dependent on n then the hypothesis is false.

\subsection{$\frac{n!}{n}$}

Assume the linear comparison algorithm exists for the number of all possible sequences (denoting $p_{1}$) have Equation \ref{eq:n1n}:
\begin{equation}
  \label{eq:n1n}
  p_{1}\geq\frac{n!}{n} =  (n-1)!
\end{equation}

Such as previous subsection, to justify the hypothesis, we need to prove \ref{eq:hvsn2}

\begin{equation}
  2^{cn} \geq 2^{h} \geq (n-1)!
  \label{eq:hvsn2}
\end{equation}

Use $lg$ function on both sides, then we'll get the induction \ref{eqa:nm1}:
\begin{eqnarray}
  \begin{split}
	2^{cn} \geq& n-1! \\
	cn \geq& lg((n-1)!) \\
	cn \geq& lg((n-1)!) \\
	cn \geq& \int_{1}^{n-2}lg(x)dx  \\
	cn \geq& (n-2)lg(n-2) - 2n +3 \\
	(c+2)n \geq& (n/2)lg(n/2) ~ ~ (n>4)\\
	(c+2)n \geq& (n/2)lg(n/2) \\
	(2c+5) \geq& lg(n)
  \end{split}
  \label{eqa:nm1}
\end{eqnarray}

From \ref{eqa:nm1}, we know that we know that c is dependent on n then the hypothesis is false.

\subsection{$\frac{n!}{2^{n}}$}

Assume the linear comparison algorithm exists for the number of all possible sequences (denoting $p_{1}$) have Equation \ref{eq:n2n}:
\begin{equation}
  \label{eq:n2n}
  p_{1}\geq\frac{n!}{2^{n}}
\end{equation}

Such as previous subsection, to justify the hypothesis, we need to prove \ref{eq:hvsn3}

\begin{equation}
  2^{cn} \geq 2^{h} \geq \frac{n!}{2^{n}}  
  \label{eq:hvsn3}
\end{equation}

Use $lg$ function on both sides, then we'll get the induction \ref{eqa:n2n}:
\begin{eqnarray}
  \begin{split}
	2^{cn} \geq& \frac{n!}{2^{n}}  \\
	cn \geq& lg(n!) - n \\
	(c+1)n \geq& \int_{1}^{n-1}lg(x)dx  \\
	(c+1)n \geq& (n-1)lg(n-1) - 2n +2 \\
	(c+3)n \geq& (n/2)lg(n/2) ~ ~ (n>1)\\
	(c+3)n \geq& (n/2)lg(n/2) \\
	(2c+7)n \geq& lg(n)
  \end{split}
  \label{eqa:n2n}
\end{eqnarray}

From \ref{eqa:n2n}, we know that we know that c is dependent on n then the hypothesis is false.

\subsection{Conclusion}
Then under these 3 circumstances, we cannot find a linear comparison algorithm.


\section{Problem 8.1-4}
The problem to solve the whole array can be divided into solve subarraries with k length.\\ For every subarraries, we know that there exists Equation \ref{equ:nlgn}
\begin{equation}
  T(k) = \Omega (klgk)
  \label{equ:nlgn}
\end{equation}
Then for any k and all the subarries, we can find a $c_{k}$ to make:
\begin{eqnarray}
  \begin{split}
	T(n)=& \frac{n}{k} T(k) +O(1)  \\
	\geq& \frac{n}{k} c_{k}klg(k) +O(1) ~ ~ (c_{k} depends ~ on ~  k) \\
	\geq& c_{k}nlg(k) +O(1) ~ \\
	=&  \Omega(nlgk) \\
  \end{split}
  \label{equa:nlgk}
\end{eqnarray}

As a conclusion, we have that $T(n) = \Omega(nlgk)$.

\end{document}
