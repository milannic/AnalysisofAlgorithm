\documentclass[oneside]{homework} %%Change `twoside' to `oneside' if you are printing only on the one side of each sheet.
\usepackage{setspace} 
\usepackage{algorithm} 
\usepackage{algorithmic}
\usepackage{multirow}
\usepackage{multicol}
\usepackage{subfigure}
\usepackage{epsfig}
\setlength{\parindent}{0cm}
\renewcommand{\algorithmicrequire}{\textbf{Require:}}
\renewcommand{\algorithmicensure}{\textbf{Iteration:}}
\renewcommand{\algorithmiclastcon}{\textbf{Output:}}
\studname{Cheng Liu}
\collaborator{Introduction to algorithms\\}
\coursename{Analysis of algorithms I}
\hwNo{6}
\uni{cl3173}
\cuni{3rd edition}
\prNo{5}
\begin{document}
\maketitle
\newpage
\section {Problem 26.1-2} 

A flow network with multiple sources and multiple sinks G is a directed graph in which each edge $(u,v) \in E$ has a nonnegative capacity $c(u,v) \geq 0$, and if E contains an edge $(u,v)$, then there is no edge $(v,u)$ in the reverse direction. \\
Then we define S be the vertex set called source set of the network, and T be the vertex set called sink set of the network,both S and T are non-empty. 

Other than vertices in S, all the vertices should have at least one entering edge, and for convenience, we assume that for each vertex $v \in V-S-T$, there is at least one pair $(s,t)(s \in S, t \in T)$, there exists at least one path from $s$ through $v$ to $t$.\\

Let G = (V,E) be a flow network with a capacity function c. \\

A flow in G is a real-valued function $f : V \times V \rightarrow \Re$ that satisfies the following two properties:

Capacity constraint : For all $u,v \in V$, we require $0\leq f(u,v) \leq c(u,v)$.

Flow conservation: For all $u \in V - S - T$, we require
$$\sum\limits_{v \in V}f(v,u) =\sum\limits_{v \in V}f(u,v) $$
When $(u,v) \notin E$, there can be no flow from $u$ to $v$, and $f(u,v) = 0$.

We call the nonnegative quantity $f(u,v)$ the flow from vertex $u$ to vertex $v$. The value $|f|$ of a flow $f$ is defined as :
$$ |f| = \sum_{s \in S}\sum_{v \in V}f(s,v) - \sum_{s \in S}\sum_{v \in V}f(v,s) $$

Then we show that any flow in a multiple-source, multiple-sink flow network corresponds to a flow of identical value in the single-source,single-sink network obtained by adding a super source (ss)and a super sink(st), and vice versa. \\
$\Rightarrow$ \\
for the original multiple-source,multiple-sink flow network, we have\\ $$ |f| = \sum_{s \in S}\sum_{v \in V}f(s,v) - \sum_{s \in S}\sum_{v \in V}f(v,s) $$ 
After the transformation, all the vertices in S have became normal vertex, and there is one (ss,s) for every s in S added to the graph.

According to flow conservation: for every $s \in S$, it should have:
$$ f(ss,s) -  \sum_{v \in V}f(s,v) - \sum_{v \in V}f(v,s) = 0$$
For the super source, the capacity is just $\infty$, then this value must be compatible.

And the  single-source,single-sink network's flow value is defined as : 
$$ |f|' = \sum_{v \in V}f(ss,v) - \sum_{v \in V}f(v,ss) = \sum_{s \in S}f(ss,s) - \sum_{s \in S}f(s,ss) $$
That is:
$$ |f|' = \sum_{s \in S}(\sum_{v \in V}f(s,v) - \sum_{v \in V}f(v,s)) = \sum_{s \in S}\sum_{v \in V}f(s,v) - \sum_{s \in S}\sum_{v \in V}f(v,s)  $$

$\Leftarrow$ \\
On the contrary, 
$$ |f|' = \sum_{v \in V}f(ss,v) - \sum_{v \in V}f(v,ss) $$
And for each edge $(ss,s)$, we just remove it and add $s$ to the S.
Recall the flow conservation:
$$ f(ss,s) + \sum_{v \in V-ss}f(v,u) = \sum_{v \in V}f(u,v)$$
After the deletion of the edge, we have:
$$\sum_{v \in V}f(u,v) - \sum_{v \in V-ss}f(v,u) =f(ss,s) $$

Then we sum up all the vertices in S, which is the flow value of multiple-source, multiple-sink network, then we have:
$$ |f| = \sum_{s \in S}\sum_{v \in V}f(s,v) - \sum_{s \in S}\sum_{v \in V}f(v,s) $$
$$ |f| = \sum_{s \in S}(\sum_{v \in V}f(s,v) - \sum_{v \in V}f(v,s)) $$
$$ |f| = \sum_{s \in S}(f(ss,s)) $$
By the super source property, we can add those terms equal to 0, then we have:
$$ |f| = \sum_{v \in V}f(ss,v)- \sum_{v \in V}f(v,ss) $$

\section {Problem 26.2-10} 
Firstly we just find out the maximum flow of the given network G = (V,E). 

Next we just go back, reduce the flow to 0. We do this in the following way. 

Firstly we scan the whole network, to find the edge$(u,v)$ has the minimal non-zero $f(u,v)$, then we just find a flow path $s \rightarrow \cdots \rightarrow u \rightarrow v \rightarrow \cdots \rightarrow t$, along the path, each edge $(u',v')$we just reduce  $ f'(u',v') = f(u',v')-f(u,v)$. Notice that we just find the minimal non-zero $f(u,v)$, and we won't break the property of the flow. 

Then we repeat above step until we get the flow to 0.

And because in each iteration, we've reduced the minimal non-zero $f(u,v)$ to 0, combining with the fact that we just perform decreasing, we will know that this edge won't be operated again.

Because $f(u,v) = 0$, $(u,v)$ will never occur in the latter flow path again, and it will never be selected as well. 
For there is $|E|$ edges, we can repeat above step at most $|E|$ times.

At last, the path by which each time we reduce the flow can be regarded as a augment path reversely if we start to find the maximum flow, then we just re-increase the flow just as the way we decrease it, so we could get to our max flow with at most $|E|$ augmentations, the sequence is just the reverse order when we decrease it.

\end{document}
