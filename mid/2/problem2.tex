\documentclass[oneside]{homework} %%Change `twoside' to `oneside' if you are printing only on the one side of each sheet.
\usepackage{setspace} 
\usepackage{algorithm} 
\usepackage{algorithmic}
\usepackage{multirow}
\renewcommand{\algorithmicrequire}{\textbf{Require:}}
\renewcommand{\algorithmicensure}{\textbf{Iteration:}}
\renewcommand{\algorithmiclastcon}{\textbf{Output:}}
\studname{Cheng Liu}
\collaborator{Introduction to algorithms\\}
\coursename{Analysis of algorithms I}
\hwNo{4}
\uni{cl3173}
\cuni{3rd edition}
\prNo{2}
\begin{document}
\maketitle
\newpage

\section*{Problem 2.} 
Because we can only visit per bikestand once, the problem is treated as a non-preemptable job schedule problem, we treat refilling a certain bikestand i as a ceratin job must start before $t_{i} - c_{i}$.

At first we say that for such a  problem if there is a possible schedule with idle time between any two neighbor jobs, we can remove those space to get a tighter scheduler and the solution is also possible, then we can always finish a job and start another job and can still find a optimum solution if there are. This shrinks our solution space.

Let B be the set of all the bikestands and S be any schedule.

{\bfseries 
  And we define the work set B compatible with t(time) is that the smallest $t_{i} - c_{i}$ of a certain job is  no larger than t. That is, it is still possible for all the jobs in the set B can be finished in time.
}

Now consider our problem E(B,t), the optimum substructure is below:(Notice that if there is possible solution, \textbf{E(B,t) = 1})
\begin{equation*}
  E(B,t) = \left \{
	\begin{array}{l}
	  {0} \qquad \text{if} \qquad \text{B is not compatible with t.}  \\
	  {1} \qquad \text{if}  B = \emptyset   \\
	  { max_{b_{i}\in B}(E(B-\{b_{i}\},t+c_{i}))} \qquad \text{else}  \\
	\end{array}
  \right \}
\end{equation*}

Now we prove the correctness of it.

1.If B is not compatible with t, it means that there is always at least a job cannot be finished after time t, thus the answer is obviously 0. 

2.If B is empty set, it means that no job should be schedule after time t, then the answer is always 1.

3.Otherwise, notice that the problem is a binary-value problem, if $E(B,t)_{op}=1$, the first job excuted is just $b_{i}$,then we take out $b_{i}$, and we say $E(B-\{b_{i}\},t+c_{i}) = E(B-\{b_{i}\},t+c_{i})_{op} = 1$ is also the optimal solution,other wise ,the E(B,t) cannnot be 1. 

If E(B,t) = 0, it means that the all the job cannot be finished, then it soon end at once.

Notice that the schedule of $E(B-\{b_{i}\},t+c_{i})$ may not be the same as $E(B-\{b_{i}\},t+c_{i})_{op}$, but for this binary-value problem, we can say $E(B-\{b_{i}\},t+c_{i})$ is also a optimum.


\textbf{The greedy algorithm is that we always take the job $b_{i}$ with the earliest $t_{i}$ in the remaining job queue to execute. }

Then we prove the safety of the greedy algorithm.

If the optimum of E(B,t) is 0, greedy algorithm will return 0,it is no doubt.

Then if the optimum of E(B,t) is 1, then we apply the greedy algorithm, we can always find a possible solution.

Notice that if 
$$\{E(B,t)_{op} = 1\} \Leftrightarrow \\ $$
$$\{ \text{there is at least one sequence of B that will let the recursion continue untill B  = } \quad \emptyset \}$$ 

$$\{E(B,t)_{SE} = 1\} \Leftrightarrow \\ $$
$$\{\text{sequence SE will let the recursion continue untill B  = } \quad \emptyset \}$$ 
Assume that $b_{i}$ is the bikestand with the earliest $t_{i}$ in B for the problem E(B,t) and the solution of greedy algorithm is $SE_{ga}$.  

Let one optimal solution sequence for E(B,t) be $SE_{op}$, if $SE_{op} = SE_{ga}$ then we have finished the proof. 

Otherwise, W.L.O.G,we assume the only difference betwwen $SE_{ga}$ and $SE_{op}$ is the position of $b_{i}$,we assume there is only $b_{k}$ before $b_{i}$ in $SE_{op}$, we can simply exchange $b_{k}$ with $b_{i}$, the new sequence is also compatible. 

Because $b_{i}$ is the job with the earliest $t_{i}$, and if in the $SE_{op}$, $b_{i}$ finishes in $f_{i}$, then there must be $f_{k} < f_{i} \leq t_{i} \leq t_{k}$, after the exchange, $f'_{i}<f'_{k} =  f_{i} \leq t_{i} \leq t_{k}$, so the new schedule must also be compatible.

Thus for our $SE_{ga}$, the recursion can also continue untill the set be empty, that is,so that is $E(B,t)_{SE_{ga}} = 1$.

Generally, if there are a lot of differences between $SE_{op}$ and  $SE_{ga}$ we can gradually exchange elements pair in $SE_{op}$ to make it $SE_{ga}$ without avoiding compatibility. 

So far we have proved that if there are schedules all the jobs can be finished then our greedy algorithm safely find one of them.

The pseudo-code is below the running time is $O(nlgn) $ dominated by the \textbf{sort}(we can apply a heap sort to this problem),the greedy algorithm only runs a length-n loop and costs $O(n)$,thus the total running time is still $O(n)$:
\begin{algorithm}[h]
  \caption{refill}
  \label{algo:refill}
  \begin{algorithmic}[1]
	\REQUIRE B,C,T
	\ENSURE ~ ~\\ 
	\STATE {sort B in the ascending order correspoding to T}
	\STATE {$let K[1..|B|] = B$}
	\STATE {$t = 0$}
	\STATE {$current = 0$}
	\WHILE {$B \neq \emptyset$ }
	  \STATE {$current = POP(B)$}
		\IF {$t > T[current] - C[current]$}
		  \STATE {$ K = \emptyset$}
		\ELSE 
		  \STATE {$ t = t+C[current] $}
		\ENDIF
	\ENDWHILE
	\LASTCON $K$
  \end{algorithmic}
\end{algorithm} 
\end{document}
