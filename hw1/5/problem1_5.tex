\documentclass[oneside]{homework} %%Change `twoside' to `oneside' if you are printing only on the one side of each sheet.
\usepackage{setspace} 
\usepackage{algorithm} 
\usepackage{algorithmic}
\usepackage{multirow}
\renewcommand{\algorithmicrequire}{\textbf{Require:}}
\renewcommand{\algorithmicensure}{\textbf{Iteration:}}
\renewcommand{\algorithmiclastcon}{\textbf{Output:}}
\studname{Cheng Liu}
\collaborator{Introduction to algorithms\\}
\coursename{Analysis of algorithms I}
\hwNo{1}
\uni{cl3173}
\cuni{3rd edition}
\prNo{5}
\begin{document}
\maketitle
\newpage
\section{Sort the function array}
\noindent The fcuntion array is ($n^2$, $n^{0.01}$, $2^n$ ,$3^{n/2}$ ,$2^{\lg \lg n}$ ,$\lg ^4 n$ ,$4^{\lg n}$ ,$16n-5 \lg n$ ,$(\lg n)^{\lg n / \lg \lg n}$ )
\\ Then we sort them by order of growth, the new array will be:$$(2^{\lg \lg n},\lg ^4 n,n^{0.01},16n-5 \lg n,(\lg n)^{\lg n / \lg \lg n} ,n^2,4^{\lg n},3^{n/2},2^n)$$
\\ Next we will prove the correctness of each pair:
\subsection*{1. ~$(2^{\lg \lg n},\lg ^4 n)$}
Let $f(n)$ denotes $2^{\lg \lg n}$ and $g(n)$ denotes $\lg ^4 n$, notice that $f(n)=2^{\lg \lg n}= \lg n$
\subsubsection* {O:}
Assume that $f(n) = O(g(n))$\\
\begin{equation}
\notag
\begin{aligned}[b]
f(n) &= O(g(n)) \\
\lg n & \leq c_{0}(\lg ^4 n)  ~ ~(2\leq n)\\
1 & \leq c_{0}(\lg ^3 n) \\
1 / (\lg ^3 n)  & \leq c_{0} \\
\end{aligned}
\end{equation}
If we take $c_{0} \geq 1$, then we can get :
$$f(n) = O(g(n))$$
\rule{3mm}{3mm}
\subsubsection* {$\Omega$:}
Assume that $f(n) = \Omega (g(n))$\\
\begin{equation}
\notag
\begin{aligned}[b]
f(n) &= \Omega(g(n)) \\
\lg n & \geq c_{1}(\lg ^4 n)  ~ ~(2\leq n)\\
1 & \geq c_{1}(\lg ^3 n) \\
1 / (\lg ^3 n)  & \geq c_{1} \\
\end{aligned}
\end{equation}
Becasue $\lim\limits_{n\to\infty}{1 / (\lg ^3 n)}  = 0$, then we cannot take a such $c_{1} > 0$.\\
\rule{3mm}{3mm}

\subsubsection* {Conclusion:}
$$2^{\lg \lg n}= O(\lg ^4 n)$$

\subsection*{2. ~$(\lg ^4 n,n^{0.01})$}
Let $f(n)$ denotes $\lg ^4 n$ and $g(n)$ denotes $ n^{0.01} $
\subsubsection* {O:}
Assume that $f(n) = O(g(n))$\\
Based on L'Hospital Rule, we can get the $\lim\limits_{n\to\infty}{n^{0.01} / (\lg ^4 n)}  = \infty$.
\\Then there must be a $n_{0}$, when $n \geq n_{0} , f(n) \leq g(n) $, then we can take $c_{0} \geq 1$.
\\ Therefore $f(n) = O(g(n))$ 
\\ \rule{3mm}{3mm}

\subsubsection* {$\Omega$:}
Assume that $f(n) = \Omega (g(n))$\\
Based on L'Hospital Rule, we can get the $\lim\limits_{n\to\infty}{n^{0.01} / (\lg ^4 n)}  = \infty$.
\\Then there won't be a pair of constant $c_{1}$ and $n_{0}$, when $n \geq n_{0} , f(n) \geq c_{1}g(n) $.
\\ \rule{3mm}{3mm}

\subsubsection* {Conclusion:}
$$2^{\lg ^4 n}= O(n^{0.01})$$


\subsection*{3. ~$(n^{0.01},16n-5 \lg n)$}
Let $f(n)$ denotes $n^{0.01}$ and $g(n)$ denotes $16n-5 \lg n$
\subsubsection* {O:}
Assume that $f(n) = O(g(n))$\\
\begin{equation}
\notag
\begin{aligned}[b]
f(n) &= O(g(n)) \\
n^{0.01} & \leq c_{0}(16n-5 \lg n)  ~ ~(0\leq n)\\
1 & \leq c_{0}(16n^{0.99}-5\lg n / n^{0.01}) \\
\end{aligned}
\end{equation}
Based on L'Hospital Rule, we can get the $\lim\limits_{n\to\infty}{n^{0.01} / (\lg ^4 n)}  = \infty $.
\\Then there must be a $n_{0}$, when $n \geq n_{0} , 5\lg n /n^{0.01} = \epsilon $, then we can take $c_{0} \geq \frac{1}{16n^{0.99}+\epsilon}$.
\\ Therefore $f(n) = O(g(n))$ 
\\ \rule{3mm}{3mm}

\subsubsection* {$\Omega$:}
Assume that $f(n) = \Omega (g(n))$\\
\begin{equation}
\notag
\begin{aligned}[b]
f(n) &= \Omega(g(n)) \\
n^{0.01} & \geq c_{1}(16n-5 \lg n)  ~ ~(0\leq n)\\
1 & \geq c_{1}(16n^{0.99}-5\lg n / n^{0.01}) \\
\end{aligned}
\end{equation}
Based on L'Hospital Rule, we can get the $\lim\limits_{n\to\infty}{n^{0.01} / (\lg ^4 n)}  = \infty$.\\
Then $$\lim\limits_{n\to\infty}{(1 / 16n^{0.99}-5\lg n / n^{0.01})}  = \infty$$
\\Then there won't be a pair of constant $c_{1}$ and $n_{0}$, when $n \geq n_{0} , f(n) \geq c_{1}g(n) $.
\\ \rule{3mm}{3mm}

\subsubsection* {Conclusion:}
$$n^{0.01} = O(16n-5 \lg n)$$

\subsection*{4. ~$(16n-5 \lg n,(\lg n)^{\lg n / \lg \lg n})$}
Let $f(n)$ denotes $16n-5 \lg n$ and $g(n)$ denotes $(\lg n)^{\lg n / \lg \lg n}$ and notice that $(\lg n)^{\lg n / \lg \lg n} = n$
\subsubsection* {O:}
Assume that $f(n) = O(g(n))$\\
\begin{equation}
\notag
\begin{aligned}[b]
f(n) &= O(g(n)) \\
16n-5 \lg n & \leq c_{0}n  
\end{aligned}
\end{equation}
\\ Just take any $ c_{0} \geq 16 $~and~$n\geq 1$, we will get $f(n) = O(g(n))$ 
\\ \rule{3mm}{3mm}

\subsubsection* {$\Omega$:}
Assume that $f(n) = \Omega (g(n))$\\
\begin{equation}
\notag
\begin{aligned}[b]
f(n) &= \Omega(g(n)) \\
16n-5 \lg n & \geq c_{1}n  
(16-c_{1})n \geq 5 \ln n
\end{aligned}
\end{equation}
Based on L'Hospital Rule, we can get the $\lim\limits_{n\to\infty} n / \lg  n  = \infty$.\\
\\ Just take any $ c_{1} \leq 16 $~and we will find a big enough $n_{0}$, we will get $f(n) = \Omega(g(n))$ 
\\ \rule{3mm}{3mm}

\subsubsection* {Conclusion:}
$$16n-5 \lg n = \Theta((\lg n)^{\lg n / \lg \lg n})$$

\subsection*{5. ~$((\lg n)^{\lg n / \lg \lg n},n^2)$}
Let $f(n)$ denotes $(\lg n)^{\lg n / \lg \lg n}$ and $g(n)$ denotes $n^2$ and notice that $(\lg n)^{\lg n / \lg \lg n} = n$
\subsubsection* {O:}
Assume that $f(n) = O(g(n))$\\
\begin{equation}
\notag
\begin{aligned}[b]
f(n) &= O(g(n)) \\
n  & \leq c_{0}n^2  
1  & \leq c_{0}n 
\end{aligned}
\end{equation}
\\ Just take any $ c_{0} \geq 1 $~and~$n\geq 1$, we will get $f(n) = O(g(n))$ 
\\ \rule{3mm}{3mm}

\subsubsection* {$\Omega$:}
Assume that $f(n) = \Omega (g(n))$\\
\begin{equation}
\notag
\begin{aligned}[b]
f(n) &= \Omega(g(n)) \\
n & \geq c_{1}n^2  
1 & \geq c_{1}n  
1/n \geq c_{1}
\end{aligned}
\end{equation}
Because $\lim\limits_{n\to\infty} 1/n  = 0$.\\
\\Then there won't be a pair of constant $c_{1}$ and $n_{0}$, when $n \geq n_{0} , f(n) \geq c_{1}g(n) $.
\\ \rule{3mm}{3mm}

\subsubsection* {Conclusion:}
$$(\lg n)^{\lg n / \lg \lg n} = \Theta(n^2)$$


\subsection*{6. ~$(n^2,4^{\lg n})$}
Let $f(n)$ denotes $n^2$ and $g(n)$ denotes $4^{\lg n}$ and notice that $g(n) = 4^{\lg n} = n^2 = f(n)$
Therefore $n^2 = \Theta (4^{\lg n})$

\subsection*{7. ~$(4^{\lg n},3^{n/2})$}
Let $f(n)$ denotes $4^{\lg n}$ and $g(n)$ denotes $3^{n/2}$ and notice that $4^{\lg n} = n^2$ and $3^{n/2} = \sqrt{3}^n$
\subsubsection* {O:}
Assume that $f(n) = O(g(n))$\\
\begin{equation}
\notag
\begin{aligned}[b]
f(n) &= O(g(n)) \\
n^2  & \leq c_{0}\sqrt{3}^n \\
n^2 / \sqrt{3}^n & \leq c_{0} \\
\end{aligned}
\end{equation}
Based on L'Hospital Rule, we can get the $\lim\limits_{n\to\infty}{n^2 / \sqrt{3}^n}  = 0$.\\
\\ Just take any $ c_{0} \geq 1 $~and~$n_{0}\geq 4$, we will get $f(n) = O(g(n))$ 
\\ \rule{3mm}{3mm}

\subsubsection* {$\Omega$:}
Assume that $f(n) = \Omega (g(n))$\\
\begin{equation}
\notag
\begin{aligned}[b]
f(n) &= \Omega(g(n)) \\
n^2  & \geq c_{0}\sqrt{3}^n \\
n^2 / \sqrt{3}^n & \geq c_{0} \\
\end{aligned}
\end{equation}
Based on L'Hospital Rule, we can get the $\lim\limits_{n\to\infty}{n^2 / \sqrt{3}^n}  = 0$.\\
\\Then there won't be a pair of constant $c_{1}$ and $n_{0}$, when $n \geq n_{0} , f(n) \geq c_{1}g(n) $.
\\ \rule{3mm}{3mm}

\subsubsection* {Conclusion:}
$$4^{\lg n} = O(3^{n/2})$$

\subsection*{8. ~$(3^{n/2},2^n)$}
Let $f(n)$ denotes $3^{n/2}$ and $g(n)$ denotes $2^n$ and notice that $3^{n/2} = \sqrt{3}^n$
\subsubsection* {O:}
Assume that $f(n) = O(g(n))$\\
\begin{equation}
\notag
\begin{aligned}[b]
f(n) &= O(g(n)) \\
\sqrt{3}^n & \leq c_{0}2^n\\
\sqrt{3}^n / 2^n & \leq c_{0} \\
(\sqrt{3} / 2)^n & \leq c_{0} \\
\end{aligned}
\end{equation}
Because $\lim\limits_{n\to\infty}{(\sqrt{3}\ 2)^n}  = 0$
\\ Just take any $ c_{0} \geq 1 $~and~$n_{0}\geq 0$, we will get $f(n) = O(g(n))$ 
\\ \rule{3mm}{3mm}

\subsubsection* {$\Omega$:}
Assume that $f(n) = \Omega (g(n))$\\
\begin{equation}
\notag
\begin{aligned}[b]
f(n) &= \Omega(g(n)) \\
\sqrt{3}^n & \geq c_{0}2^n\\
\sqrt{3}^n / 2^n & \geq c_{0} \\
(\sqrt{3} / 2)^n & \geq c_{0} \\
\end{aligned}
\end{equation}
Because $\lim\limits_{n\to\infty}{(\sqrt{3}\ 2)^n}  = 0$
\\Then there won't be a pair of constant $c_{1}$ and $n_{0}$, when $n \geq n_{0} , f(n) \geq c_{1}g(n) $.
\\ \rule{3mm}{3mm}

\subsubsection* {Conclusion:}
$$3^{n/2} = O(2^n)$$
\end{document}
