\documentclass[oneside]{homework} %%Change `twoside' to `oneside' if you are printing only on the one side of each sheet.
\usepackage{setspace} 
\usepackage{algorithm} 
\usepackage{algorithmic}
\usepackage{multirow}
\renewcommand{\algorithmicrequire}{\textbf{Require:}}
\renewcommand{\algorithmicensure}{\textbf{Iteration:}}
\renewcommand{\algorithmiclastcon}{\textbf{Output:}}
\studname{Cheng Liu}
\collaborator{Introduction to algorithms\\}
\coursename{Analysis of algorithms I}
\hwNo{2}
\uni{cl3173}
\cuni{3rd edition}
\prNo{1}
\begin{document}
\maketitle
\newpage

\section*{Exersice 8.3-4}

\subsection* {1.Regular Part}
At first, let we introduce Lemma \ref{lemma:introtextbook}:\\
\begin{lemma}
  Given n b-bit numbers and any positive integer 
  \begin{math}
	r\leq b
  \end{math}
  , Radix-Sort correctly sorts these numbers in 
  \begin{math}
	\Theta((b/r)(n+2^{r}))
  \end{math}
  time if the stable sort it uses 
  \begin{math}
	\Theta(n+k) 
  \end{math}
  time for the inputs ranges from 0 to k-1.
  \label{lemma:introtextbook}
\end{lemma}

For any integer $k$ between 0 to $n^{3}-1$, we treat it as a binary number with $lg(n^{3}) = 3\lceil lg(n) \rceil$ bit, as for Lemma \ref{lemma:introtextbook}, we take 
\begin{equation}
  \begin{split}
	b = 3\lceil lg(n) \rceil , 
	r = \lceil lg(n) \rceil
  \end{split}
  \label{equ:denoting1}
\end{equation}
For $\lceil lg(n) \rceil$ bits binary number, it ranges from 0 to $n-1$ in decimal. \\
Then if we can sort them in $\Theta(n+n) = \Theta(n)$ time,
according to Lemma \ref{lemma:introtextbook}, we can sort these n numbers in $\Theta((\frac{\lceil 3lg(n) \rceil}{\lceil lg(n) \rceil}(n+2^{\lceil lg(n) \rceil}))) = \Theta (3(n+n)) = \Theta(n)$.

As is well known, we can easily find an algorithm such as counting sort to stable sort n numbers ranging from 0 to  $n-1$, thus the procedure to solve the problem is modified-radix sort, that is we take each $lg(n)$ binary bit as a individual part of each number ranging from 0 to $n^{3}-1$ to execute the radix sort.

\subsection*{2. Compared to Heapsort}
As we all know, Heapsort is a kind of comparison sort method, then we have 
\begin{equation}
  \begin{split}
	T_{HeapSort}(n) = \Theta(n \lg n)
  \end{split}
  \label{equ:hs}
\end{equation}

Gradually expand the range of n numbers mentioned above, then we will find a lower bound for the range that Radix-Sort method cannot solve it in liear time and must spend $\Theta (n \lg n)$ time on it.
Intuitively, as the division method mentioned above, the range from 0 to $n^{c}-1$ for any constant $c$ can be solved by Radix-Sort in linear time.

Then we set c to $\lg n$, the number of the input will be from 0 to $n^{\lg n}-1$, then we divide them per $lgn$ bits. As we mentioned above, each $lgn$ bits can also be sorted within $\Theta(n)$ time.  \\
While this time, the coefficient $b/r$ will be $\lg {n^{\lg n}} / \lg n  = \lg n$
Then the whole method cost 
\begin{equation}
  \Theta(n\lg n) 
  \label{equ:nlg}
\end{equation}

We change another division method to set $ r = \frac{\lg n^{\lg n}}{2}$ (the constant coefficient 2 can be any positive integer),
the coefficient $b/r$ will be $2$, but we need $\Theta (n+ 2^{\lg n^{\lg n}})$ to sort every part of the numbers and the total cost will be: 
\begin{equation}
  \Theta(n^{\lg n}) 
  \label{equ:nplgn}
\end{equation}
And this result will be larger than previous division.
Then as a conclusion, when we set the range as from 0 to $n^{\lg n}-1$, the radix sort method will have the same performance as the Heap Sort.

\end{document}
