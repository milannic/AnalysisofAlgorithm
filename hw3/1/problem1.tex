\documentclass[oneside]{homework} %%Change `twoside' to `oneside' if you are printing only on the one side of each sheet.
\usepackage{setspace} 
\usepackage{algorithm} 
\usepackage{algorithmic}
\usepackage{multirow}
\renewcommand{\algorithmicrequire}{\textbf{Require:}}
\renewcommand{\algorithmicensure}{\textbf{Iteration:}}
\renewcommand{\algorithmiclastcon}{\textbf{Output:}}
\studname{Cheng Liu}
\collaborator{Introduction to algorithms\\}
\coursename{Analysis of algorithms I}
\hwNo{3}
\uni{cl3173}
\cuni{3rd edition}
\prNo{1}
\begin{document}
\maketitle
\newpage

\section*{Problem 9.4}
\subsection* {a.}
At first we define that: 
\\  $X_{ijk}$  = I \{ $z_{i}$ is compared with $z_{j}$ sometime during the execution of the algorithm to find $z_{k}$ \}
\\If we want $X_{ijk} = 1$, we must ensure that before we pick up the kth statistic, $z_{i}$ or $z_{k}$ have been selected as the pivot.
%\\Then we define that $Y_{ijk} = l$ means that $z_{j}$ and $z_{i}$ is compared in the lth random-partition.\\
\\When $i < k < j$, if there is no elment in $[i,j]$ selected as the pivot, $z_{i}$ or $z_{j}$ is still possible to do a comparation. 
\\Thus P($X_{ijk}$ = 1) = P($z_{i}$ or $z_{j}$ is the first elements selected as pivot in $[i,j]$.
Then:
\begin{equation}
  E[X_{ijk}] = P(X_{ijk}=1) = \frac{2}{j-i+1}
  \label{equ:ikj}
\end{equation}
%$$P(Y_{ijk} = 1) = 2*\frac{1}{n}$$
%Similarily, in the second run, the possible reminder length is that:
When we have $i < j \leq k$, this time ,$z_{i}$ or $z_{j}$ must be the first element to be selected as pivot in $[i,k]$.
\begin{equation}
  E[X_{ijk}] = P(X_{ijk}=1) = \frac{2}{k-i+1}
  \label{equ:ijk}
\end{equation}
Similarily, when $ k \leq i < j $
\begin{equation}
  E[X_{kij}] = P(X_{kij}=1) = \frac{2}{j-k-1}
  \label{equ:kij}
\end{equation}

\subsection*{b.}

\begin{equation} 
  \begin{aligned}
	E[X_k] &= \sum_{i=1}^{n-1}\sum_{j=i+1}^{n}E[X_{ijk}] \\
   & \leq \sum_{i \leq k \leq j} \frac{2}{j - i + 1} + \sum_{i<j<k} \frac{2}{k - i + 1} + \sum_{k<i<j} \frac{2}{j - k + 1} \\
	     & = \sum_{i = 1}^k \sum_{j = k}^n \frac{2}{j - i + 1} + \sum_{i = 1}^{k - 2} \sum_{j = i + 1}^{k - 1} \frac{2}{k - i + 1} + \sum_{i = k + 1}^{n - 1} \sum_{j = i + 1}^{n} \frac{2}{j - k + 1} \\
	  & = 2(\sum_{i = 1}^k \sum_{j = k}^n \frac{1}{j - i + 1} + \sum_{j = k + 1}^{n} \frac{j - k - 1}{j - k + 1} + \sum_{i = 1}^{k - 2} \frac{k - i - 1}{k - i + 1})
  \end{aligned}
  \label{equ:b}
\end{equation}

\subsection*{c.}
Firstly notice that in Equation \ref{equ:b}, for last two term, we have:
\begin{eqnarray}
  \begin{split}
	\sum_{j = k + 1}^{n} \frac{j - k - 1}{j - k + 1} + \sum_{i = 1}^{k - 2} \frac{k - i - 1}{k - i + 1} &\leq \sum_{j = k + 1}^{n}1 + \sum_{i = 1}^{k - 2}1 \\
	 &\leq \sum_{i = 1}^{n}1 \\
	 &= n \\
  \end{split}
\end{eqnarray}
Then we check the first term :
\begin{eqnarray}
  \begin{split}
	\sum_{i = 1}^k \sum_{j = k}^n \frac{1}{j - i + 1}
  \end{split}
\end{eqnarray}
For  i = 1, we have:
\begin{eqnarray}
  \begin{split}
	\sum_{j = k}^n \frac{1}{j} = \frac{1}{k} + \frac{1}{k+1} + \frac{1}{k+2} + \cdots + \frac{1}{n}
  \end{split}
\end{eqnarray}
For  i = 2, we have:
\begin{eqnarray}
  \begin{split}
	\sum_{j = k}^n \frac{1}{j-1} = \frac{1}{k-1} + \frac{1}{k} + \frac{1}{k+1} + \cdots + \frac{1}{n-1}
  \end{split}
\end{eqnarray}

\begin{eqnarray*}
  \cdots
\end{eqnarray*}
For  i = k, we have:
\begin{eqnarray}
  \begin{split}
	\sum_{j = k}^n \frac{1}{j-k+1} = 1 + \frac{1}{2} + \frac{1}{3} + \cdots + \frac{1}{n-k+1}
  \end{split}
\end{eqnarray}

\begin{equation}
  \mathbf{\Phi}=\left[
 \begin{array}{ccccc}
   \frac{1}{k} & \frac{1}{k+1}	& \frac{1}{k+2} & \cdots & \frac{1}{n} \\
   \frac{1}{k-1} & \frac{1}{k}	& \frac{1}{k+1} & \cdots & \frac{1}{n-1} \\
  \vdots & \vdots & \vdots & \vdots & \vdots \\
   \frac{1}{2} & \frac{1}{3}	& \frac{1}{4} & \cdots & \frac{1}{n-k+2} \\
   1 & \frac{1}{2}	& \frac{1}{3} & \cdots & \frac{1}{n-k+1} \\
\end{array}
\right]
\label{equ:Gmatrixnumber}
\end{equation}
We can see from the matrix above, the summation of every diagnoal is just no larger than 1. \\
Thus the whole value of the matrix is no larger than n.
\\
\begin{equation} 
  \begin{aligned}
	E[X_k] 	  &\leq 2(\sum_{i = 1}^k \sum_{j = k}^n \frac{1}{j - i + 1} + \sum_{j = k + 1}^{n} \frac{j - k - 1}{j - k + 1} + \sum_{i = 1}^{k - 2} \frac{k - i - 1}{k - i + 1})
	  &\leq 2(n +n)
	  &\leq 4n
  \end{aligned}
  \label{equ:f}
\end{equation}

\subsection* {d.}
As we know,the total cost of Randomized-Select in every recursive level is due to Randomized-Partition with some constant cost, and the max recursion level is just $n$, thus the total cost by constant cost is only $O(n)$.
As for Randomized-Partition, the total cost of all Randomized-Partition in different recursion level is just the $E[X_{k}]$ with constant coefficient, we have prove that 
$E[X_{k}] \leq 4n$, thus the total cost of a Randomized-Select is:
\begin{eqnarray}
  \begin{split}
	T(n) &= O(c_{1}n + c_{2}4n)
	 &= O(n)
  \end{split}
\end{eqnarray}
\end{document}
