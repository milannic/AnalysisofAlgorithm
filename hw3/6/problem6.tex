\documentclass[oneside]{homework} %%Change `twoside' to `oneside' if you are printing only on the one side of each sheet.
\usepackage{setspace} 
\usepackage{fancybox}
\usepackage{tikz}
\usepackage{algorithm} 
\usepackage{algorithmic}
\usepackage{multirow}
\renewcommand{\algorithmicrequire}{\textbf{Require:}}
\renewcommand{\algorithmicensure}{\textbf{Iteration:}}
\renewcommand{\algorithmiclastcon}{\textbf{Output:}}
\studname{Cheng Liu}
\collaborator{Introduction to algorithms\\}
\coursename{Analysis of algorithms I}
\hwNo{3}
\uni{cl3173}
\cuni{3rd edition}
\prNo{6}
\begin{document}
\maketitle
\newpage
\section {Problem 15-12}
At first we sort the position and assign each player into their position array, then we give some notations,V[i,j],C[i,j] each means the VORP and cost of the $j$th free-agent player in position i.\\
And $S_{op}(i,X)$ means the optimum set of player from first i positions and the total maximum budget is X.
Then we need to find the optimal sub-structure in the problem. And the optimum sub-structure is that:\\
1.if there is a free-agent player in position i in $S_{op}(i,X)$, denoting it as $P_{i,o}$ and its index in VORP and cost table is $(i,j_{io})$, then we exclude $P_{i,o}$ from 
$S_{op}(i,X)$ to make a new set S', $S' = S_{op}(i-1,X-C[i,j_{io}])$.
\\2.if there is no certain free-agent player in position i in $S_{op}(i,X)$, then $S_{op}(i,X) = S_{op}(i-1,X)$.

It is easy to prove this.
Let's assume that $S_{op}(i,X)$ contain a $P_{i,o}$, but 
$$S_{op}(i-1,X-C[i,j_{io}]) > S'(i-1,X-C[i,j_{io}]) = S_{op}(i,X)-P_{i,o}$$
And $ S_{op}(i-1,X-C[i,j_{io}] $ is different from $S'(i-1,X-C[i,j_{io}])$ in at least one position selection, assuming $P_{i-1,o}$, then we change our plan to make $S_{op}(i-1,X-C[i,j_{io}] + P_{i,o} = S'(i,X) > S_{op}(i,X)$ \\ 
Thus our original plan is not the optimum, this conflicts with our hypothesis.
Similarily, when $S_{op}(i,X)$ contain no $P_{i,o}$, we can prove the sub-structure.
\\[4px]
As for the pseudo-code, we first arrange our player data to build V[i,j],C[i,j],P[i,j] table, every P players in the same position(such as i) are set to the sub-array,V[i,:](VORP) and C[i,:](Cost). \\And we know that the unit of price is 100,000, then we define a 2-D array T with the dimension of [N+1,X/100,000+1].\\ 
There is a three level nested loop in the Algorithm \ref{algo:dp}, firstly we check each position,next we check every player in this position, then for every possible money,we find the $S_{op}(i,m)$.\\ 
In the beginning of the most outside loop, we just initialize the $T[i,0 \cdots mlength] = T[i-1,0 \cdots mlength]$, if there is T[i,m] doesn't change in this loop, it means that $S_{op}(i,m)$ doesn't contain any player in position i.\\
Thus the time complexity of the algorithm is $\Theta(XNP)$ and the space complexity is $\Theta(NX)$

\begin{algorithm}[h]
  \caption{Sign-Free-Player-FINDSOLUTION}
  \label{algo:dp}
  \begin{algorithmic}[1]
	\REQUIRE Data,X,N
	\ENSURE ~ ~\\ 
	\STATE{From Data Build Table V[i,j] and C[i,j]}
	\STATE{moneyunit = 100,000}
	\STATE{mlength = X/moneyunit}
	\STATE{Let $T[0 \cdots N,0 \cdots mlength]$ be the new array with all the elements are 0}.
	\STATE{Let $K[0 \cdots N,0 \cdots mlength]$ be the new array with all the elements are 0}.
	\FOR {$i = 1$ to N}
	\STATE {$T[i,0 \cdots mlength] = T[i-1,0 \cdots mlength]$}
	  \FOR {$j=1$ to P}
		\FOR {$ m =1$ to mlength}
		  \IF {$C[i,j]/moneyunit <= m$}
			\IF {$V[i,j]+T[i-1,m-C[i,j/moneyunit] > T[i,m]$}
			  \STATE {$T[i,m] = V[i,j]+T[i-1,m-C[i,j/moneyunit]$}
			  \STATE {$K[i,m] = j$}
			\ENDIF
		  \ENDIF
		\ENDFOR
	  \ENDFOR
	\ENDFOR
	\STATE{ $count = mlength$}
	\FOR{$pos = n$ to 1}
	  \IF{$K[pos,money] \neq 0$}
		\STATE{PRINT ( pos,V[pos,K[pos,count]],C[pos,K[pos,count]] )}
		\STATE{$count= count- C[pos,K[pos,count]$}
	  \ENDIF
	\ENDFOR
	\LASTCON T[N,mlength]
  \end{algorithmic}
\end{algorithm}

\end{document}
